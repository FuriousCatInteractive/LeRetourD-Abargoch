\documentclass[12pt]{article}

\usepackage[utf8]{inputenc}
\usepackage[T1]{fontenc}
\usepackage[french]{babel}
\usepackage[top=2cm, bottom=2cm, left=2cm, right=2cm]{geometry}
\usepackage{listings}
\lstset{
	frame=single,
	rulesep=1mm,
	framesep=5mm,
	framerule=1pt,
	xrightmargin=5mm,
	xleftmargin=5mm,
	language=Java,
	basicstyle=\footnotesize,
	numbers=left,
	numberstyle=\normalsize,
	numbersep=7pt,
}

\usepackage{graphicx}

\title{High Level Concept}
\author{Steven \bsc{Gerard}, Corentin \bsc{Raoult}, Loic \bsc{Tessier}}
\date {14 Novembre 2014}

\begin{document}
\maketitle{}

\section{Présentation}
Il s’agira d’un "tower defense" en 2D vue de dessus avec une part de jeu de rôle. Ce jeu
s'inspirera du jeu "Orc must die". C'est à dire que le joueur aura une phase où il aura le temps
de poser des pièges, des bonus etc. Puis une phase de combat durant laquelle le joueur pourra
lui-même se battre contre les monstres en s'aidant des pièges posés durant la phase précédente.

\section{Caractéristiques}
\subsection{Partie Tower Defence}
\subsubsection{Déroulement de la phase de jeu}

\section{Vue d'ensemble}
\subsection{Motivations du joueur}
Le ou les joueurs choisissent un personnage avec certaine caractéristiques propre et doivent essayer de repousser les vagues d'ennemis. Ceci afin gagner de l'or et des points d'expérience prmettant d'améliorer leur personnage au fur et à mesure de la partie.
\subsection{genre}
Il s'agit d'un "tower defense" avec une phase action/jeu de rôle.
\subsection{public visée}
Ce jeu s'ardresse à des personnes ayant un minimum d'expérience video-ludique mais sans être des "hardcores gamers".
\subsection{plateforme hardware}
A cause des contrôles à la manette, le jeu s'oriente plus vers les consoles de salon et les PC.
\subsection{Caractéristiques uniques}
Mélange des genres Jeux de rôle, tower defense et beat'em all
tower defense coopératif
\subsection{Compétition}
Ce jeu se joue en coopération et n'est pas en réseau, il n'est donc pas orienté "e-sport"/compétition.


	
\end{document}
